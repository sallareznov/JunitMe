\section{Validation}
\label{sec:Validation}
	Our project has been tested on *****, a Java open-source project ***** (GitHub link). We fixed the number of generated tests to 10000, and measured the code coverage by varying the alloy run count. The tool used to calculate the coverage is EclEmma.\cite{eclemma}\\
	
\begin{figure}[!h]
\begin{center}
ADD LOC \\~\\
\begin{tabular}{|c|c|}

 \hline
 \textbf{Project name} & \textbf{Coverage percentage}\\
 \hline
 asdfqsdqfsdf   & 80\%\\ 
 \hline
 vqsfsdqfsdq & 84\%\\
 \hline
 dsfsdfsdqfsdqf & 90\%\\

 \hline
\end{tabular}
\caption{Code coverage of generated tests with the Alloy run count fixed to 8}
\end{center}
\end{figure}

The increase of the Alloy scope improved the tests code coverage. The results we obtained are represented in the following plot :\\

\pgfplotsset{width=\textwidth,compat=1.9}

{\setlength{\parindent}{0cm}
\begin{tikzpicture}
\begin{axis}[
	axis lines=left,
    title={Code coverage of generated tests},
    xlabel={Alloy run count},
    ylabel={Code coverage (\%)},
    xmin=0, xmax=12,
    ymin=0, ymax=100,
    ytick={0,20,40,60,80,100},
    xtick={0,2,4,6,8,10},
    ymajorgrids=true
]
 
\addplot[
    color=red,
    mark=circle,
    ]
    coordinates {
    (0,0)(1, 10)(2, 20)(3, 30)(4, 40)(5, 50)(6, 60)(7, 70)(8, 80)(9, 90)
    };
 
\end{axis}
\end{tikzpicture}
}