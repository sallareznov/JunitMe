\section{Validation}
\label{sec:Validation}
	Our project has been tested on three projects from the SF100 benchmark \cite{sf100} : Jipa \cite{jipa}, a simple interpreter in Java, Inspirento \cite{inspirento}, a nice journal maker, and Newzgrabber \cite{newzgrabber}, an application used for retrieving file attachments to newsgroup articles. We fixed the number of generated tests to 10000, and measured the code coverage by varying the alloy run count. The tool used to calculate the coverage is EclEmma.\cite{eclemma}\\
	
\begin{figure}[!h]
\begin{center}
\begin{tabular}{|c|c|}

 \hline
 \textbf{Project name} & \textbf{Coverage percentage}\\
 \hline
 Jipa & 82.9\%\\ 
 \hline
 Inspirento & 78.0\%\\
 \hline
 Newzgrabber & 87.2\%\\

 \hline
\end{tabular}
\caption{Code coverage of generated tests with the Alloy run count fixed to 8 and the number of tests fixed to 10000}
\end{center}
\end{figure}

The increase of the Alloy run count improved the tests code coverage. For example, the relation between the Alloy run count and the percentage of covered code for the project Jipa (which is a small project) is represented in the following plot :\\

\pgfplotsset{width=\textwidth,compat=1.9}

{\setlength{\parindent}{0cm}
\begin{tikzpicture}
\begin{axis}[
	axis lines=left,
    title={Code coverage of generated tests for the Alloy run count},
    xlabel={Alloy run count},
    ylabel={Code coverage (\%)},
    xmin=0, xmax=14,
    ymin=0, ymax=100,
    ytick={0,20,40,60,80,100},
    xtick={0,2,4,6,8,10,12},
    ymajorgrids=true
]
 
\addplot[
    color=red,
    mark=square,
    ]
    coordinates {
    (0,0)(2,36.4)(4,48.5)(6,55.6)(8,68.7)(10,76.7)(12,82.9)
    };
 
\end{axis}
\end{tikzpicture}
}