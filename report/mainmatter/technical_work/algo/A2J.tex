\subsection{Alloy to Java}
\subsubsection{Generating Java unit tests}
\paragraph{Modelizing Alloy instances in Java}
Using Alloy Analyzer we execute the generated Alloy source code. Alloy Analyzer generates all possible instances.
Each instance is a solution, we can obtain a solution using A4Solution object. A4Solution object has a method \textit{satisfiable} to check if the solution is valid and a method \textit{next} to go the next possible solution.

\paragraph{Creating Java object}
Each alloy instance is represented by a Java object called: \textit{AlloyInstance}.***************

\paragraph{Generating tests}
In order To generate the code of Java unit tests we used CodeModel, which allows us to generate Java classes in a simple way.\\
We browse Java modelized solution and we generate the variables used to call methods and the variables passed as methods parameters.\\
For each solution we use the execution trace.
Firstly, we initalize all the necessary types for the receiver method, then all the veriables that will be used in parameters.

\begin{algorithm}[H]
\SetAlgoLined
\KwData{Alloy instance modelized in Java}
\KwResult{Java unit test}
 \ForEach{Alloy Instance modelized in Java}{
		Get the head of method calls list\;
		\While{ methods calls are not finished}{
 		 Create method signature\;
 		 Add $org.Junit.Test$ annotation \;
		 Add $Throw Exception$ to themethod signature\;
		 Create a $try catch$ block\;
		 Inside a $try$ block:\{\\
	     - Declare all variables\\
	     - Call constructors\\
	     - Add method calls with the necessary parameters\\
		 \}\;
 	}}
\caption{How to transfer an Alloy instance modelized in Java to Java unit test}
\end{algorithm}
\bigskip
\bigskip
An example of the results output:
\lstset{ %
  backgroundcolor=\color{white},   % choose the background color
  basicstyle=\footnotesize,        % size of fonts used for the code
  breaklines=true,                 % automatic line breaking only at whitespace
  captionpos=b,                    % sets the caption-position to bottom
  commentstyle=\color{mygreen},    % comment style
  escapeinside={\%*}{*)},          % if you want to add LaTeX within your code
  keywordstyle=\color{blue},       % keyword style
  stringstyle=\color{mymauve},     % string literal style
}
\begin{lstlisting}[language=java]
@Test
public void testmethod() {
  try{
  }catch(java.lang.Exception exception){
  }
}
\end{lstlisting}