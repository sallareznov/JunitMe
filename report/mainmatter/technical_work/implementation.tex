\section{Implementation}
\label{sec:Implementation}
In order to realise our project we have used four different techonologies: Spoon, Alloy, Alloy Analyzer and CodeModel.\\
The main parts of our project is generating Alloy source code from a given Java program, then generating Alloy instances. Thereafter modelizing these instances in Java to, finally, convert these modelized instances to Java unit tests.

\subsubsection{Spoon}
\label{subsec:impl}
In order to analyze and transform source code, we needed an efficient and powerful library.
We have chosen Spoon, a high-quality open-source library created and maintained by INRIA
(French Institute for Research in Computer Science and Automation (French: Institut national
de recherche en informatique et en automatique)). Providing a complete and fine-grained
JAVA metamodel, Spoon enables us to perform effortless treatments in differents parts of code.
These treatments are performed by processors, which are able to browse, modify, or even add
any program element (class, method, field, statements, expressions...).\cite{spoon-gforge}\cite{spoon-hal}
Spoon can also be used on validation purpose, to ensure that your programs respect some
programming conventions or guidelines, or for program transformation, by using a pure-JAVA
template engine.\cite{spoon-javasource}
\subsubsection{Alloy}
Alloy is a language for describing structures and a tool for exploring them. It has been used in a wide range of applications from finding holes in security mechanisms to designing telephone switching networks.
An Alloy model is a collection of constraints that describes (implicitly) a set of structures, for example: all the possible security configurations of a web application, or all the possible topologies of a switching network.\cite{alloy}

\subsubsection{Alloy Analyzer}
Alloy Analyzer, is a solver that takes the constraints of a model and finds structures that satisfy them. It can be used both to explore the model by generating sample structures, and to check properties of the model by generating counterexamples.\cite{alloy}

\subsubsection{CodeModel}
CodeModel is a Java library for code generators; it provides a way to generate Java programs in a way much nicer than PrintStream.println().
With CodeModel, we can build the java source code by first building AST (Abstract syntax tree)\cite{ast}, then writing it out as text files that is Java source files.\cite{codeModel}