\section{Implementation \& architecture}
\label{sec:Implementation}
In order to realise our project we have used four different techonogies: Spoon, Alloy, Alloy Analyzer and CodeModel.\\
The main parts of our project is generating Alloy source code from a given Java program, then generating Alloy instances. Thereafter modelizing these instances in Java to, finally, convert these modelized instances to Java unit tests.

\subsubsection{Spoon}
In order to analyze and transform source code, we needed an efficient and powerful library.
We have chosen Spoon, a high-quality open-source library created and maintained by INRIA
(French Institute for Research in Computer Science and Automation (French: Institut national
de recherche en informatique et en automatique)). Providing a complete and fine-grained
JAVA metamodel, Spoon enables us to perform effortless treatments in differents parts of code.
These treatments are performed by processors, which are able to browse, modify, or even add
any program element (class, method, field, statements, expressions...).\cite{spoon-gforge}\cite{spoon-hal}
Spoon can also be used on validation purpose, to ensure that your programs respect some
programming conventions or guidelines, or for program transformation, by using a pure-JAVA
template engine.\cite{spoon-javasource}
\subsubsection{Alloy}
Alloy is a language for describing structures and a tool for exploring them. It has been used in a wide range of applications from finding holes in security mechanisms to designing telephone switching networks.
An Alloy model is a collection of constraints that describes (implicitly) a set of structures, for example: all the possible security configurations of a web application, or all the possible topologies of a switching network.\cite{alloy}

\subsubsection{Alloy Analyzer}
Alloy Analyzer, is a solver that takes the constraints of a model and finds structures that satisfy them. It can be used both to explore the model by generating sample structures, and to check properties of the model by generating counterexamples.\cite{alloy}

\subsubsection{CodeModel}
CodeModel is a Java library for code generators; it provides a way to generate Java programs in a way much nicer than PrintStream.println().
With CodeModel, we can build the java source code by first building AST (Abstract syntax tree)\cite{ast}, then writing it out as text files that is Java source files.\cite{codeModel}


\subsubsection{Generating Alloy source code}
\paragraph{Base model}
The base model is a meta-model, It is the base of the generated java to Alloy program.
It is composed of four parts: Types, methods, methods constraints and arguments of these methods.


\paragraph{Code generation}
To generate Alloy instances for an existing Java program we used Spoon. Spoon uses AST (Abstract syntax tree)\cite{ast} to browse the structure of a specified program. With informations collected in the AST, we are able to generate the code of Alloy program corresponding to the initial Java program. The generated code is then grafted to the base model.
Result is a complet Alloy model inside the file FinalGen.als

\subsubsection{Generating Java unit tests}
\paragraph{Modelizing Alloy instances in Java}
Using Alloy Analyzer we execute the generated Alloy source code. Alloy Analyzer generates all possible instances.
Each instance is a solution, we can obtain a solution using A4Solution object. A4Solution object has a method \textit{satisfiable} to check if the solution is valid and a method \textit{next} to go the next possible solution.

\paragraph{Creating Java object}
Each alloy instance is represented by a Java object called: \textit{AlloyInstance}.***************

\paragraph{Generating tests}
In order To generate the code of Java unit tests we used CodeModel, which allows us to generate Java classes in a simple way.\\
We browse Java modelized solution and we generate the variables used to call methods and the variables passed as methods parameters.\\
For each solution we use the execution trace.
Firstly, we initalize all the necessary types for the receiver method, then all the veriables that will be used in parameters.